\documentclass[a4wide, 11pt]{article}
\usepackage{a4, fullpage}
\setlength{\parskip}{0.2cm}
\setlength{\parindent}{0cm}
\usepackage[hmargin=2.2cm,vmargin=2.3cm]{geometry}
\usepackage{url}
\usepackage{hyperref}
\usepackage{breakurl}

\ifx\pdftexversion\undefined
\usepackage[dvips]{graphicx}
\else
\usepackage[pdftex]{graphicx}
\DeclareGraphicsRule{*}{mps}{*}{}
\fi

\begin{document}

\title{WebApps Group Project - Integrated Event Management and Discovery System for members of Imperial College London}

\author{Harry Lachenmayer, Artur Spychaj, Alex Rozanski and Thomas Rooney}

\date{\today}         % inserts today's date

\maketitle            % generates the title from the data above

\section {Introduction}

\subsection {Requirements}

The product we are building is meant to be a tool to aid college members in discovering new events. To do this, we had several requirements that we agreed upon in the projects conception.

\begin{itemize}
\item Authenticate with the college systems - no need to register. We consider this incredibly important because to attract a userbase from nothing we need to minimise the amount of interaction the user would need to do with the app to recieve useful information.
\item Provide automatic access to \textit{at least} the feeds that are already provided, such as the Imperial College Union \`Whats On\' calendar. We wanted to provide value without user interaction, so that it can be used immediately as an information source.
\item Provide user interactivity to provide information into the system. While the website as-is is intended to be a strong information source, the hope is that it provides an easier way to advertise events to members of the college, and that it could be a preferred platform for contacting people and cause a snowball effect in userbase increase. This requires that there are mechanisms available to add events.
\end{itemize}

\subsection {Target}

When we first began talking about the application, we were very careful to ensure that we carefully went through each of the potential user categories of the application. We were limited to members of the College, but we split these down into the following characters, and provided user stories for how we wanted to provide value for each of them.

\begin{itemize}
\end{itemize}

\section {Project Management}	

\subsection {Group Structure}

\subsection {Implementation Technology Decisions}
\subsubsection {Node.js}
\subsubsection {Coffeescript}
\subsubsection {Nginx}
\subsubsection {Neo4j}

\subsection {Design Processes}
\subsection {Version Control System}

\section {System Description}

\section {System Implementation}
\subsection {Front End}
\subsection {Back End}

\section {External Packages}

package.json

\section {Conclusion}



\end{document}
